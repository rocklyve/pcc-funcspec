\documentclass[a4paper,twoside,BCOR=20mm]{scrreprt}
\usepackage[utf8]{inputenc}
\usepackage[T1]{fontenc}
\usepackage[ngerman]{babel}	% german hyphenation, quotes, etc
\usepackage[ngerman]{translator}
\usepackage{amsmath}
\usepackage{paralist}
\usepackage{amsfonts}
\usepackage{acronym}
\usepackage{enumerate}
\usepackage{hyperref}
\usepackage{amssymb}
\usepackage{caption}
\usepackage{multirow}
\usepackage{graphicx}
\usepackage{tabularx}
\usepackage{color}
\usepackage{wrapfig} % wrap text around figures
\usepackage{subfig} % align two pics beside each other
\usepackage[table,xcdraw]{xcolor}
\hypersetup{ 					% ‘texdoc hyperref‘ for options
	pdftitle={PSE PCC: Pflichtenheft},
	pdfauthor={Giorgio Groß, Christoph Hörtnagl, David Laubenstein,  Josh Romanowski,  Fabian Wenzel},
	bookmarks=true,
}
\title{Pflichtenheft: Crash- Cam}

%Paket laden
\usepackage[
numberedsection,
nonumberlist, %keine Seitenzahlen anzeigen
acronym,      %ein Abkürzungsverzeichnis erstellen
toc,          %Einträge im Inhaltsverzeichnis
section]      %im Inhaltsverzeichnis auf section-Ebene erscheinen
{glossaries}

%Befehle für Abkürzungen
\newacronym{KIT}{KIT}{Karlsruher Institut für Technologie}

%Richtige Silbentrennungen
\hyphenation{Ein-stel-lun-gen}
\hyphenation{Be-stä-ti-gungs-di-a-log}

% KIT layout

\definecolor{orange}{rgb}{1,0.5,0}
\definecolor{mintgreen}{RGB}{50,161,137}
\definecolor{gray}{RGB}{120,120,120}

\usepackage[color]{changebar}
\cbcolor{gray}
\changebarwidth 0.5pt

\usepackage{fancyhdr}
\pagestyle{fancy}
 \fancyhf{} %alle Kopf- und Fußzeilenfelder bereinigen 
 
 \fancypagestyle{plain}{} %Kopf- und Fußzeile auf jeder Seite	 
	\fancyhead[L]{Pflichtenheft}
	\fancyhead[R]{\leftmark}
	\rhead{\nouppercase{\leftmark}}
	\renewcommand{\headrulewidth}{0.5pt}
	\renewcommand{\headrule}{\hbox to\headwidth{%
		\color{mintgreen}\leaders\hrule height \headrulewidth\hfill}}

\raggedbottom

	\renewcommand{\footrulewidth}{0.5pt}
	\renewcommand{\footrule}{\hbox to\headwidth{%
  		\color{mintgreen}\leaders\hrule height \headrulewidth\hfill}}				
	\fancyfoot[LE,RO]{\thepage}


\setcounter{tocdepth}{5}
\makeglossaries
\begin{document}
\begin{titlepage}

\begin{center}

\includegraphics[width=0.5\linewidth]{subtopicsFuncspec/Res/KITLogo.png}\\[0.5cm]
  

\textsc{\bfseries Fraunhofer Institut für Optronik, Systemtechnik und Bildauswertung}\\[0.5cm]
\textsc{Mario Kaufmann\\Pascal Birnstill\"uckl}\\[2cm]

\textsc{\LARGE \bfseries Pflichtenheft}\\[0.5cm]
\textsc{\bfseries Version 0}\\[0.2cm]


\newcommand{\HRule}{\rule{\linewidth}{0.5mm}} 
{\color{mintgreen}\HRule} \\[0.4cm]
{\huge \bfseries Privacy Dash Cam App für Android}\\[0.4cm]
{\color{mintgreen}\HRule} \\[1cm]

% \textsc{\Large \bfseries Gruppe :}\\[0.3cm] 
\textsc{\Large Giorgio Groß\\Christoph Hörtnag\\David Laubenstein\\Josh Ramanowski\\Fabian Wenzel} \\[2cm]

{\large \today}

\end{center}

\end{titlepage}
% \maketitle
\tableofcontents
\newpage
%content
\chapter{Zielbestimmung}

\includegraphics[width=\textwidth]{subtopicsFuncspec/Res//Mockups/Portrait_camera_view_car.jpg}\\[0.5cm]

Das Produkt ermöglicht es seinen Nutzern ihre Autofahrten zu überwachen, indem es durch die \gls{Smartphone}kamera den Straßenverkehr verfolgt und relevantes Videomaterial \glslink{persistieren}{persistent} abspeichert. Dieses wird bei Bedarf verwendet, um Unfallhergänge im Straßenverkehr zu dokumentieren. Dabei gilt es, dem deutschen Datenschutzrecht gerecht zu werden, indem Personen und personenbezogene Daten, wie zum Beispiel Autokennzeichen, unkenntlich gemacht werden. Die Anwendung bietet dem Nutzer dafür eine moderne und intuitive Bedienoberfläche.\newline
Zur Realisierung kann das Produkt in drei Hauptbestandteile aufgegliedert werden: Die \gls{Android} \gls{App}, den \gls{Web-Dienst} und das \gls{Web-Interface}. Diese Aufteilung wird in diesem Heft, um eine Übersicht zu schaffen auch in nachfolgenden Kapiteln beibehalten.

\section{Musskriterien}
\subsection{App}
	\begin{enumerate}
	\renewcommand{\labelenumi}{\textbf{\theenumi}}
	\renewcommand{\theenumi}{PK\arabic{enumi}0}
	\setcounter{enumi}{99}
	\item Nutzer müssen sich anmelden, um die \gls{App} zu verwenden.
	\item Nur registrierte Nutzer können die \gls{App} verwenden.
	\item Der Straßenverkehr wird durch die \gls{Smartphone}kamera beobachtet.
	\item Relevante Videodaten werden verschlüsselt abgespeichert.
	\item Relevante Daten werden durch Auswertung der Sensordaten des \glspl{Smartphone} erkannt. Hierbei werden die Werte des G-Sensors ausgewertet.
	\item Während der Aufnahme werden sämtliche Nutzereingaben und \gls{G-Sensor}daten ignoriert.
	\item Es werden relevante \gls{Metadaten} mit den Videodaten abgespeichert.
	\item Es wird ab dem \gls{App}start mit dem Beobachten des Straßenverkehrs begonnen.
	\item Es wird nur das Hochformat unterstützt.
	\item Die Beobachtung läuft nur während sich die \gls{App} im Vordergrund befindet.
	\item Videodaten werden verschlüsselt, sobald sie \glslink{persistieren}{persistiert} werden.
	\item Verschlüsselte Videodaten werden aufgelistet.
	\item Verschlüsselte Videodaten können gelöscht werden.
	\item Vom Nutzer ausgewählte verschlüsselte Videodaten werden an einen \gls{Web-Dienst} gesendet, der diese \glslink{anonymisieren}{anonymisiert}.
	\item Geräte, auf denen \gls{Android} \gls{API} Level 19 (Android 4.4) und höher läuft werden unterstützt.
	\item Die Benutzeroberfläche wird für Geräte ab einer Displaydiagonale von 4 Zoll optimiert.
	\item Wenn verschlüsselte Videodaten lange Zeit nicht zum \glslink{anonymisieren}{Anonymisieren} ausgewählt wurden, wird der Nutzer benachrichtigt, dass er diese löschen kann.
	\item Die aufnahmespezifischen Einstellungen werden angezeigt.
	\item Die Standardsprache ist Deutsch.
	\end{enumerate}
\subsection{Web-Dienst}
	\begin{enumerate}
	\renewcommand{\labelenumi}{\textbf{\theenumi}}
	\renewcommand{\theenumi}{PK\arabic{enumi}0}
	\setcounter{enumi}{199}
	\item Es existiert eine Schnittstelle, um Videodaten hochzuladen.
	\item Von der \gls{App} gesendete Videodaten werden \glslink{anonymisieren}{anonymisiert}.
	\item Nach Abschluss der \glslink{anonymisieren}{Anonymisierung} wird der Nutzer per \gls{E-Mail} benachrichtigt.
	\item Es existiert eine Schnittstelle, um Nutzeraccounts anzulegen.
	\item Es existiert eine Schnittstelle, um Nutzeraccounts zu verwalten.
	\item Es existiert eine Schnittstelle, um die Videodaten eines Nutzers verwalten zu können.
	\item Nutzer müssen ihre \gls{E-Mail}-Adresse verifizieren, um sich anmelden zu können.
	\item Die Kommunikation zwischen App und Web-Dienst wird durch eine REST-\gls{API} realisiert.
	\item Die Kommunikation zwischen \gls{Web-Interface} und \gls{Web-Dienst} wird durch eine REST-\gls{API} realisiert.
	\item Es existiert eine obere Schranke für die Anzahl der Videodaten, die ein Nutzer zur gleichen Zeit auf seinem Nutzeraccount online speichern kann.
	\item Passwörter werden nur als \gls{Hash-Code} abgespeichert.
	\item Es wird Jetty verwendet.
	\end{enumerate}
\subsection{Web-Interface}
	\begin{enumerate}
	\renewcommand{\labelenumi}{\textbf{\theenumi}}
	\renewcommand{\theenumi}{PK\arabic{enumi}0}
	\setcounter{enumi}{299}
	\item Es können Nutzeraccounts angelegt werden.
	\item Es können Nutzeraccounts verwaltet werden.
	\item Es können Videodaten verwaltet werden.
	\item Es können Videodaten heruntergeladen werden.
	\item Nur eingeloggte Nutzer haben Zugriff auf ihre Nutzerdaten.
	\item Es können Passwort und \gls{E-Mail}-Adresse geändert werden.
	\item Die Standardsprache ist Deutsch.
	\end{enumerate}

\section{Wunschkriterien}
\subsection{App}
	\begin{enumerate}
	\renewcommand{\labelenumi}{\textbf{\theenumi}}
	\renewcommand{\theenumi}{WK\arabic{enumi}0}
	\setcounter{enumi}{99}
	\item Falls es \gls{Android} zulässt, wird die Beobachtung auch ausgeführt, wenn sich die \gls{App} im Hintergrund befindet.
	\item Sowohl Quer- als auch Hochformat werden unterstützt.
	\item Die Beobachtung kann manuell gestartet und gestoppt werden.
	\item Die Aufnahme kann manuell gestartet werden, auch wenn der \gls{G-Sensor} des Smartphones keinen Anlass dazu gibt.
	\item Während der Aufnahme wird eine Möglichkeit angeboten, die Aufnahme abzubrechen.
	\item Es können Nutzeraccounts angelegt werden.
	\item Es können Nutzeraccounts verwaltet werden.
	\item \glslink{anonymisieren}{Anonymisierte} Videodaten können vom Server gelöscht werden.
	\item Push-Benachrichtignungen vom \gls{Web-Dienst} werden angezeigt.
	\item Es können Hilfestellungen für die Bedienung der \gls{App} angezeigt werden.
	\item \glslink{anonymisieren}{Anonymisierte} Videodaten können heruntergeladen werden.
	\item \glslink{anonymisieren}{Anonymisierte} Videodaten die zum Download bereit stehen und sich nicht auf dem \gls{Smartphone} befinden werden aufgelistet.
	\item \glslink{anonymisieren}{Anonymisierte} Videodaten, die sich auf dem \gls{Smartphone} befinden, werden aufgelistet.
	\item \glslink{anonymisieren}{Anonymisierte} Videodaten können vom Speicher des \glspl{Smartphone} gelöscht werden.
	\item \glslink{anonymisieren}{Anonymisierte} Videodaten, die sich auf dem \gls{Smartphone} befinden, können angesehen werden.
	\item Die aufnahmespezifischen Einstellungen können bearbeitet werden.
	\item So lange der \gls{G-Sensor} des Smartphones Anlass zur Aufnahme gibt wird weiter aufgenommen.
	\end{enumerate}
\subsection{Web-Dienst}
	\begin{enumerate}
	\renewcommand{\labelenumi}{\textbf{\theenumi}}
	\renewcommand{\theenumi}{WK\arabic{enumi}0}
	\setcounter{enumi}{199}
	\item Die Zeit, die die \glslink{anonymisieren}{Anonymisierung} in Anspruch nehmen wird, wird geschätzt und an den Nutzer weitergeleitet.
	\item Es wird eine Push-Benachrichtigung an das \gls{Smartphone} gesendet, sobald die Anonymisierung der Videodaten abgeschlossen ist.
	\item Videodaten werden maximal vier Wochen gespeichert.
	\item Der Nutzer erhält eine Woche bevor ein Video glöscht wird eine \gls{E-Mail}-Banchrichtigung.
	\end{enumerate}
\subsection{Web-Interface}
	\begin{enumerate}
	\renewcommand{\labelenumi}{\textbf{\theenumi}}
	\renewcommand{\theenumi}{WK\arabic{enumi}0}
	\setcounter{enumi}{299}
	\item Das Produkt wird neuen Nutzern präsentiert.
	\item Nutzer können ihre anonymiserten Videos online ansehen.
	\end{enumerate}

\section{Abgrenzungskriterien}
	\begin{enumerate}
	\renewcommand{\labelenumi}{\textbf{\theenumi}}
	\renewcommand{\theenumi}{AK\arabic{enumi}0}
	\setcounter{enumi}{99}
	\item Das Betrachten nicht \glslink{anonymisieren}{anonymisierter} Videodaten ist nicht möglich.
	\item Videodaten werden nicht automatisch persistiert, sobald die Beobachtung läuft.
	\item \glspl{Livestream} werden nicht unterstützt.
	\item Die \glslink{anonymisieren}{Anonymisierung} findet nicht auf dem \gls{Smartphone} statt.
	\item Der \gls{Web-Dienst} speichert Videodaten nicht auf unbegrenzte Zeit und in unbegrenzter Anzahl.
	\item Vom Speicher des \glslink{Smartphone}{Smartphones} werden Videodaten nicht automatisch gelöscht.
	\item Von Nutzern zurückgelegte Wege und besuchte Orte werden nicht aufgezeichnet.
	\item Die \gls{App} ist nicht mit \gls{Windows Phone} und \gls{iOS} kompatibel.
	\item Die \gls{App} wird nicht für \glspl{Tablet} optimiert.
	\item Ein Account ist nicht für die Nutzung durch mehrere Nutzer gedacht.
	\end{enumerate}
\chapter{Produkteinsatz}

\section{Zielgruppe}
Die Privacy Dash Cam verfolgt zwei grundlegende Ziele: Die handelsüblche Dash Cam zu ersetzen und ihren Einsatz in Deutschland so rechtskonform wie möglich zu machen. Daraus lässt sich eine eindeutige Zielgruppe ableiten, die sich aus in Deutschland ansässigen Personen die midestens  Führerscheinklasse B vorweisen können zusammensetzt. Dabei haben sowohl Vielfahrer als auch Gelegenheitsfahrer Bedarf an der Privacy Dash Cam. Es ist somit zu erwarten, dass sämtliche Nutzer 17 Jahre oder älter sein werden und ein großer Teil ein eigenes Gefährt besitzen wird. Darüber hinaus müssen Nutzer ein Smartphone besitzen, welches den Geräteanforderungen der App gerecht wird. Dies schränkt das Alter der Zielgruppe auf etwa 70 Jahre nach oben hin ein. Kenntnisse über die Benutzung des Smartphones, Verständins der deutschen Sprache und ein Internetzugang sind weitere Voraussetzungen für die Verwendung der App.

\section{Einsatz}
Nachdem die App auf einem kompatiblen Smartphone installiert und ein Nutzeraccount erstellt wurde, findet sie ihren Einsatz vorwiegend im Straßenverkehr. Der Nutzer platziert dazu sein Smartphone mithilfe einer speziellen Halterung an der Frontscheibe seines Gefährtes und ermöglicht so der Kamera ein deutliches Bild  auf die Straße vor ihm. Davon abgesehen wird sie, nachdem relevantes Videomaterial aufgezeichnet wurde, verwendet, um besagtes Material dem Webservice zum anonymisieren zuzusenden. Im Gegenzug wird sie auch verwendet, um irrelevantes Videomaterial zu löschen.\newline
Das Web-Interface stellt die zweite Instanz dar, mit der der Nutzer direkt interagieren kann. Als diese kommt sie zum Einsatz, sobald der Nutzer anonymisiertes Videomaterial oder seinen Account verwalten möchte.


\section{Anwendungsbereiche} ??

\chapter{Produktumgebung}
\begin{tabularx}{\textwidth}{|X|X|X|X|}
\hline
\rowcolor[HTML]{C0C0C0} ~ & {\textbf{\gls{App}}} & {\textbf{\gls{Web-Interface}}} & {\textbf{\gls{Web-Dienst}}} \\ \hline
\cellcolor[HTML]{C0C0C0} \multirow{4}{*}{} {\textbf{Software}} & \multirow{3}{*}{}\gls{Android} Version 19 (KitKat 4.4) oder höher & Betriebssystem mit Internetverbindung & Linux-Betriebssystem (Debian 8) \\ \cline{3-4} 
\cellcolor[HTML]{C0C0C0}    ~ & ~ & \multirow{2}{*}{}Browser (min):
\begin{itemize}
\item Google Chrome 23
\item Safari 6
\item Mozilla Firefox 17
\end{itemize} & Java \gls{Web-Dienst} (min):
\begin{itemize}
\item Jetty, v5.0
\item Jersey, v2.x
\end{itemize} \\ \cline{4-4}
\cellcolor[HTML]{C0C0C0}    ~ & ~ & ~ & Java in Version 8  \\ \cline{4-4}
\cellcolor[HTML]{C0C0C0}    ~ & ~ & ~ & Datenbank: PostgreSQL in Version 9.5.5 \\ \hline 
\cellcolor[HTML]{C0C0C0}    {\textbf{Hardware}} & \gls{Android} \gls{Smartphone} mit: 
\begin{itemize}
\item \gls{G-Sensor},
\item Kamera,
\item Internet-Verbindung,
\end{itemize}
& Computer mit Betriebssystem und Internetverbindung & Ein aus dem Internet erreichbarer Server \\ \hline
\end{tabularx}
\section{Funktionale Anforderungen}

\subsection{Kundenverwaltung (Platzhalter)}

\subsection{Seminarverwaltung (Platzhalter)}

\subsection{Rechnung erstellen (Platzhalter)}
\section{Produktdaten}

\subsection{Kundendaten (Platzhalter)}

\subsection{Seminardaten (Platzhalter)}

\subsection{Buchungsdaten (Platzhalter)}
\chapter{Nichtfunktionale Anforderungen}

\section{App}
\begin{enumerate}[\bfseries{NA}10]
\setcounter{enumi}{99}
\item \textbf{Ringpufferkapazität} \hfill\\ Der Ringpuffer speichert XX Minuten an Video-Daten.

\item \textbf{Zusatzspeicher nach Auslösen} \hfill\\ Nach Auslösen des G-Sensors/Speicherbutton wird zusätzlich zum Ringpuffer die folgenden XX Minuten Video persistiert.

\item \textbf{G Sensor Empfindlichkeit frontal} \hfill\\  Der G-Sensor soll bei einer Vorwärts-/Rückwärts-Bewegung beim überschreiten von 3 G auslösen.

\item \textbf{G Sensor Empfindlichkeit horizontal} \hfill\\  Der G-Sensor soll bei einer Links-/rechts Bewegung beim Überschreiten von 3 G auslösen.

\item \textbf{G Sensor Empfindlichkeit vertikal} \hfill\\  Der G-Sensor soll bei einer Auf-/Ab Bewegung beim Überschreiten von 4 G auslösen.

\item \textbf{Videokapazität der App} \hfill\\  Ein Benutzer kann bis zu 20 Videos in der App speichern und verwalten.

\item \textbf{Benachrichtigung zum Löschen} \hfill\\  Die Benachrichtigung zum Löschen eines Videos, soll 10 Tage nach dessen Speicherung erfolgen.
\end{enumerate}

\section{Web-Dienst}
\begin{enumerate}[\bfseries{NA}10]
\setcounter{enumi}{199}

\item \textbf{Parallele Zugriffe} \hfill\\  Es sollen bis zu XX Videos parallel anonymisiert werden können.

\item \textbf{Anonymisierungsdauer} \hfill\\  Ein Anonymisierungsvorgang soll nicht länger als XX Sekunden dauern.
\end{enumerate}

\section{Website}
\begin{enumerate}[\bfseries{NA}10]
\setcounter{enumi}{299}

\item \textbf{Größenanpassung} \hfill\\  Die Webseite soll Responsive-Design umsetzen.

\item \textbf{Gleichzeitiges Benutzen der Website} \hfill\\  Es sollen XX Benutzer in der Lage sein, die Website gleichzeitig zu benutzen.

\item \textbf{Benutzerkontenkapazität} \hfill\\  Es sollen bis zu XX Benutzerkonten verwaltet werden können.

\item \textbf{Videokapazität der Website} \hfill\\  Ein Benutzer kann bis zu 10 Videos auf der Webseite speichern und verwalten.

\item \textbf{Benachrichtigung zum Löschen} \hfill\\  Die Benachrichtigung zum Löschen eines Videos, soll 15 Tage nach dessen Speicherung erfolgen.

\item \textbf{Zeit die Videos gespeichert bleiben} \hfill\\  Videos sollen nicht länger als 30 Tage gespeichert werden.
\end{enumerate}






\chapter{Globale Testf\"alle}
\section{Erkl\"arung zu den Testsuites der Qualit\"atssicherung}
Um eine m\"oglichst fl\"achendeckende Qualita\"atssicherung zu ermo\"oglichen, werden wir im Folgenden zwischen verschiedenen Arten von Testabla\"aufen unterscheiden. Wir bieten automatisierte Tests, die das Backend und Frontend testen. Das Ziel ist, 80 Prozent des geschriebenen Codes ausf\"urhrlich zu testen. \\
Zudem werden manuelle Tests durchgef\"uhrt, 
Wir teilen die Testphase in mehrere Teilphasen ein, um Struktur in das Testen zu bekommen.
\subsection{Komponententests}
Komponententests sind allgemein dazu da, um einzelne Komponenten der Software zu testen. In unserem Beispiel werden wir den Java-Server, die Android-App und das Webinterface auf deren Funktionalit\"at testen. 
\subsection{Integration-Testsuite}
In der Integration-Testsuite wird die Kommunikation der Komponenten getestet.
\subsection{Systemtests}
Die Software wird nun auf einer realen Umgebung installiert und mit Testdaten gef\"ullt. Dort wird die Software unter realen Bedingungen getestet.

\section{Testzsenarien}
\subsection{REST-Request des Webinterfaces}
Der Kunde f\"uhrt auf der Anmeldeseite einen Log-In durch udn will nun seine gespeicherten Videos verwalten. Dabei wird ein REST-Request des Webinterface an den Server gesendet, dieser bearbeitet den Req\"ust und liefert als Response eine Antwort, die dann vom Webinterface dargestellt wird.

\subsection{Speichere gerade aufgenommenen Unfall}
Der G-Sensor des Ger\"ats l\"ost aus, da gerade ein Unfall geschehen ist. Nun wird das Video verschl\"usselt auf dem lokalen Speicherbereich abgelegt. Zudem wird es nun unter "noch nicht hochgeladene" Unf\"alle angezeigt. Die M\"oglichkeit zum hochladen, l\"oschen und umbenennen des Unfalls wird angezeigt und jede der M\"oglichkeiten funktioniert.

\subsection{Anonymisierung des Videos auf dem Java-Server}
Der Kunde hat einen Unfall aufgenommen und diesen lokal gespeichert. Nun bet\"atigt er die Funktion "Unfall hochladen". Das bereits lokal verschl\"usselte Video wird zum Server gesendet und dieser legt die Datei in das tempor\"ares Verzeichnis ab. Das Video wird anonymisiert und richtig auf dem Server abgelegt. Dem Usereintrag in der Datenbank wird um den Pfad zum neu abgespeicherten Video erg\"anzt, damit der Kunde das Video nun auch auf seiner App/Webinterface sehen kann.

\chapter{Systemmodelle}
\section{Anwendungsfälle}
\section{Bedienung der Android App}
\begin{center}
\includegraphics[width=1\textwidth]{subtopicsFuncspec/systemModels/AppAWFDiagram.png}
\end{center}
Dieser Anwendungsfall beschreibt die Bedienung der App. 
Der Benutzer kann hierbei mehrere Aktionen ausführen:
\begin{description}
\item Login durchführen
\item Videos verwalten
\item Den Status setzen
\item Manuelle Speicherung beginnen
\end{description}
Um auf alle Funktionen zugreifen zu können muss sich der Benutzer zunächst auf der App einloggen. 
Wenn der Benutzer sein aufgenommenes Videomaterial verwalten will kann er zum einen die Videos auf den Web-Service hochladen oder die Videos löschen.
Des weiteren kann er den Status der „Aufnahme“ auf aktiv/inaktiv setzen. Bei aktiv ist der Ringpuffer angeschaltet. 
Außerdem kann man einen manuellen Speichervorgang ohne des Auslösen des Sensors beginnen um zum Beispiel.

\section{Bedienung der Website}
\begin{center}
\includegraphics[width=1\textwidth]{subtopicsFuncspec/systemModels/WebsiteAWFDiagram.png}
\end{center}
Dieser Anwendungsfall beschreibt die Bedienung der Website.
Der Benutzer kann hierbei mehrere Aktionen ausführen:
\begin{description}
\item Login durchführen
\item Registrieren
\item Benutzerdaten ändern
\item Videos runterladen
\end{description}
Um den Service der App und der Website zu nutzen muss sich der Benutzer zu Beginn mit seinen Benutzerdaten registrieren. 
Nun kann er den Login auf der Website oder App durchführen um die Funktionen beider Applikationen zu nutzen. 
Ist der Benutzer eingeloggt kann er seine Benutzerdaten ändern und bereits hochgeladene Videos nach der Anonymisierung runterladen.
Nach der Registrierung und Änderung von Benutzerdaten schickt der Web-Service eine Verifizierungsmail an den Benutzer.
\chapter{Entwicklungsumgebung}

\section{Entwicklungstools}

\begin{flushleft}
\begin{tabularx} {\textwidth}{|X|X|} \hline
\gls{Android} IDE & Android Studio \\ \hline
Java \gls{IDE} & IntelliJ IDEA \\ \hline
Projektmanagement & Atlassian JIRA \\ \hline
Textverarbeitung & LaTeX \\ \hline
TeX-Distribution & TexLive \\ \hline
LaTeX Editor & TexMaker \\ \hline
UML Tool & Umlet \\ \hline
Versionskontrolle & Git \\ \hline
\end{tabularx}
\end{flushleft}

\section{Verwendete Technologien}

\begin{flushleft}
\begin{tabularx} {\textwidth}{|X|X|} \hline
Programmiersprache (\gls{App}, \gls{Web-Dienst} und \gls{Web-Interface}) & Java 8 \\ \hline
Web-Framework & Vaadin 7 \\ \hline
\gls{Java-Servlet} und Http Server & Jetty 9.3.14 \\ \hline
Serverkommunikation & RESTful \\ \hline
RESTful-Framework & Jersey 2.24.1 \\ \hline
Datenbank & PostgreSQL \\ \hline
Videobearbeitung & OpenCV \\ \hline
\end{tabularx}
\end{flushleft}

\section{Beschreibung}

\begin{description}
\item \textbf{\gls{Android} Studio} \hfill \\
Für die Implementierung der \gls{Android} \gls{App} wird die offizielle \gls{Android}-Entwicklungsumgebung \gls{Android} Studio von Google verwendet.

\item \textbf{IntelliJ IDEA} \hfill \\
IntelliJ IDEA ist eine Java Entwicklungsumgebung, die zusätzlich zu dem üblichen Umfang anderer gängiger \glspl{IDE} Support für die Entwicklung mit Vaadin, Jetty und Jersey anbietet.

\item \textbf{Atlassian JIRA} \hfill \\
Atlassian JIRA bietet eine Webanwendung zur Projektverwaltung. Dort werden Aufgaben erfasst, verwaltet und dokumentiert.

\item \textbf{LaTeX} \hfill \\
Um eine einfache, einheitliche und stabile Formatierung zu gewährleisten wird zur Texterstellung LaTeX  anstelle klassischer Texteditoren wie Word verwendet. Umgesetzt wird dies durch die Tex-Distribution TexLive und den Editor TexMaker.

\item \textbf{Umlet} \hfill \\
Zum einfachen Entwerfen von UML-Diagrammen wird das Tool Umlet verwendet.

\item \textbf{Git} \hfill \\
Git bietet ein teamfähiges (nicht-lineares) Versionskontrollsystem an, über das alle Daten des Projekts erfasst werden.

\item \textbf{Java} \hfill \\
Da alle verwendeten Technologien auf Java basieren, verwenden wir für alle Module (\gls{App}, \gls{Web-Dienst} und \gls{Web-Interface}) Java.

\item \textbf{Vaadin} \hfill \\
Für die Realisierung des \gls{Web-Interface} wird Vaadin verwendet. Vaadin ermöglicht die Weboberfläche vollständig in Java zu schreiben und bietet moderne responsive Layouts an.

\item \textbf{Jetty} \hfill \\
Für den \gls{Web-Dienst} und das \gls{Web-Interface} läuft auf dem Server Jetty. Jetty bietet eine Kombination aus \gls{Java-Servlet} und Http Server.

\item \textbf{RESTful} \hfill \\
Um Anfragen zwischen den einzelnen Modulen zu vereinheitlichen verwenden wir die Kommunikationsschnittstelle RESTful.

\item \textbf{Jersey} \hfill \\
Für die Umsetzung eines RESTful \gls{Web-Dienst}es in Java wird das Framework Jersey verwendet.

\item \textbf{PostgreSQL} \hfill \\
Zur Verwaltung der Nutzerdaten und der hochgeladen Videos wird das Datenbanksystem PostgreSQL eingesetzt.

\item \textbf{OpenCV} \hfill \\
Zur Erkennung der zu anonymisierenden Bildbereiche, sowie zur Anwendung der Anonymisierungsfilter werden OpenCV Algorithmen verwendet.

\end{description}
\chapter{Anhang}
\printglossaries
\newglossaryentry{Test}
{
  name = Test,
  plural = Tests,
  description = {Ein Test ist ein Test zum testen, blaaaaaaaaaaa},
}
%end content

\end{document}

