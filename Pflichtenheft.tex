\documentclass[11pt,a4paper]{article}
\usepackage[utf8]{inputenc}
\usepackage[german]{babel}
\usepackage{amsmath}
\usepackage{amsfonts}
\usepackage{hyperref}
\usepackage{amssymb}
\usepackage{graphicx}
\usepackage[left=2cm,right=2cm,top=2cm,bottom=2cm]{geometry}
\author{Giorgio Groß, Christoph Hörtnagl, David Laubenstein,  Josh Ramanowski,  Fabian Wenzel}
\title{Pflichtenheft: Crash- Cam}
\begin{document}
\maketitle
\newpage
\tableofcontents
\newpage
\section{Zielbestimmung}


\subsection{Musskriterien}

\subsection{Wunschkriterien}

\subsection{Abgrenzungskriterien}

\section{Produkteinsatz}

\subsection{Anwendungsbereiche}

\subsection{Zielgruppen}

\subsection{Betriebsbedingungen}


\section{Produktumgebung}

\subsection{Software}

\subsection{Hardware}

\subsection{Orgware (Netzwerkverbindung zum...)}

\subsection{Produktschnittstellen}


\section{Funktionale Anforderungen}

\subsection{Kundenverwaltung (Platzhalter)}

\subsection{Seminarverwaltung (Platzhalter)}

\subsection{Rechnung erstellen (Platzhalter)}



\section{Produktdaten}

\subsection{Kundendaten (Platzhalter)}

\subsection{Seminardaten (Platzhalter)}

\subsection{Buchungsdaten (Platzhalter)}


\section{Nichtfunktionale Anforderungen}


\section{Globale Testfälle}


\section{Systemmodelle}


\subsection{Szenarien}


\subsection{Anwendungsfälle}

\subsection{Objektmodelle}

\subsection{Dynamische Modelle}

\subsection{Benutzerschnittstellen - (Bildschirmskizzen, Navigationspfade, etc (Platzhalter))}


\section{Glossar}

\end{document}