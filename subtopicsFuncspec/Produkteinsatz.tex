\chapter{Produkteinsatz}

\section{Zielgruppe}
Die Privacy Dash Cam verfolgt zwei grundlegende Ziele: Die handelsüblche Dash Cam zu ersetzen und ihren Einsatz in Deutschland so rechtskonform wie möglich zu machen. Daraus lässt sich eine eindeutige Zielgruppe ableiten, die sich aus in Deutschland ansässigen Personen die midestens  Führerscheinklasse B vorweisen können zusammensetzt. Dabei haben sowohl Vielfahrer als auch Gelegenheitsfahrer Bedarf an der Privacy Dash Cam. Es ist somit zu erwarten, dass sämtliche Nutzer 17 Jahre oder älter sein werden und ein großer Teil ein eigenes Gefährt besitzen wird. Darüber hinaus müssen Nutzer ein Smartphone besitzen, welches den Geräteanforderungen der App gerecht wird. Dies schränkt das Alter der Zielgruppe auf etwa 70 Jahre nach oben hin ein. Kenntnisse über die Benutzung des Smartphones, Verständins der deutschen Sprache und ein Internetzugang sind weitere Voraussetzungen für die Verwendung der App.

\section{Einsatz}
Nachdem die App auf einem kompatiblen Smartphone installiert und ein Nutzeraccount erstellt wurde, findet sie ihren Einsatz vorwiegend im Straßenverkehr. Der Nutzer platziert dazu sein Smartphone mithilfe einer speziellen Halterung an der Frontscheibe seines Gefährtes und ermöglicht so der Kamera ein deutliches Bild  auf die Straße vor ihm. Davon abgesehen wird sie, nachdem relevantes Videomaterial aufgezeichnet wurde, verwendet, um besagtes Material dem Webservice zum anonymisieren zuzusenden. Im Gegenzug wird sie auch verwendet, um irrelevantes Videomaterial zu löschen.\newline
Das Web-Interface stellt die zweite Instanz dar, mit der der Nutzer direkt interagieren kann. Als diese kommt sie zum Einsatz, sobald der Nutzer anonymisiertes Videomaterial oder seinen Account verwalten möchte.


\section{Anwendungsbereiche} ??
