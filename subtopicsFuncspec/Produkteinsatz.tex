\chapter{Produkteinsatz}

\section{Zielgruppe}
Die Privacy Dash Cam verfolgt zwei grundlegende Ziele: Eine Dash Cam für das Smartphone anzubieten und ihren Einsatz mit dem deutschen Datenschutzrecht in Einklang zu bringen. Daraus lässt sich eine eindeutige Zielgruppe ableiten, die sich aus in Deutschland und Auto-, LKW und Motorrad-Fahrern zusammensetzt. Dabei haben sowohl Viel- als auch Gelegenheitsfahrer Bedarf an der Privacy Dash Cam. Darüber hinaus müssen Nutzer ein Smartphone besitzen, welches den Geräteanforderungen der App gerecht wird. Weiterhin werden Kenntnisse über die Benutzung des Smartphones, Verständins der deutschen Sprache und ein Internetzugang für die Verwendung des Produktes vorrausgesetzt.

\section{Einsatz}
Nachdem die App auf einem kompatiblen Smartphone installiert und ein Nutzeraccount erstellt wurde, findet sie ihren Einsatz im Straßenverkehr. Der Nutzer platziert dazu sein Smartphone mithilfe einer speziellen Halterung an der Frontscheibe seines Kraftfahrzeugs und ermöglicht so der Kamera ein deutliches Bild auf die Straße vor ihr. Darüber hinaus wird das Smartphone nach dem Aufzeichnen von relevantem Videomaterial verwendet, um besagtes Material dem Webservice zum anonymisieren zu senden.\newline
Das Web-Interface stellt die zweite Instanz dar, mit der der Nutzer direkt interagieren kann. Mit Hilfe dieser kann der Nutzer seinen Account verwalten und das von Web-Service anonymisierte Videomaterial herunterladen.

