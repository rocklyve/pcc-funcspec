\chapter{Anhang}
\printglossaries
\newglossaryentry{Android}
{
  name = Android,
  description = {Android ist eine von der Open Handset Alliance entwickletes Betriebssystem, sowie auch eine Software Platform für mobile Geräte(\gls{Smartphone} und \gls{Tablet}). Dabei handelt es sich um eine freie und quelloffene Software, die auf einem Linux-Kernel basiert},
}
\newglossaryentry{App}
{
  name = App,
  plural = Apps,
  description = {Als Applikationen, oder eher als Kurzform App bekannt, wird eine Anwendungssoftware auf einem mobilen Betriebssystem bezeichnet. Bei mobilen Applikationen wird zwischen verschiedenen Typen unterscheidet. Zum einen existieren native Apps (Platformabhängig) und Web-, Hybrid- und Cross-Plattform-Apps (Platformunabhängig)},
}
\newglossaryentry{API}
{
  name = API,
  description = {Eine API (Application Programming Interface), beziehungsweise Programmierschnittstelle, macht es einfacher ein Programm zu entwickeln, da es bestimmte Funktionsblöcke zur Verfügung stellt. Diese werden mit einem Softwaresystem mitgeliefert und stellen eine Programmieranbindung auf Quelltext-Ebene dar.},
}
\newglossaryentry{Web-Dienst}
{
  name = Web-Dienst,
  plural = Web-Dienste,
  description = {Bei Webdiensten (oder auch Webservice), handelt es sich um Softwarebausteine, die auf verschiedenen Netzwerkrechnern laufen und über das Internet zu einer Anwendung verbunden werden (Quelle Profi4Project.com, 19.07.2002). Dabei werden meist Maschine-zu-Maschine-Interaktionen bereitgestellt.},
}
\newglossaryentry{Web-Interface}
{
  name = Web-Interface,
  description = {Ein Web-Interface (auch Web-Schnittstelle), bezeichnet eine Schnittstelle zu einem System, die über das Hypertext Transfer Protocol (HTTP), angesprochen wird. Dabei wird unterschieden zwischen einer grafischen Benutzerobfläche und einem \gls{Webdienst}, durch das mit einem anderen System interagiert werden kann},
}
\newglossaryentry{G-Sensor}
{
  name = G-Sensor,
  description = {Der G-Sensor (bekannt als Beschleunigungssensor), ist ein Sensor der die Beschleunigung in verschiedene Bewegungsrichtungen misst. Durch bestimmte Techniken kann mit dem Sensor eine Geschwindkeitszu- oder abnahme detektiert werden.},
}
\newglossaryentry{Metadaten}
{
  name = Metadaten,
  description = {Metadaten enthalten Informationen über andere Dateien, aber nicht die Dateien selbst. Diese Informationen werden bei der Ansammlung größerer Datenmengen (wie Dokumente, Datenbanken oder Dateien) benötigt, um diese zu verwalten.},
}
\newglossaryentry{persistieren}
{
  name = persistieren,
  description = {Die Persistenz (Verb: persistieren) wird in der Informatik häufig als "nicht flüchtige Datenspeicherung" definert. Damit meint man die Möglichkeit, Daten oder Objekte über längere Zeit bereitzuhalten},
}
\newglossaryentry{anonymisieren}
{
  name = anonymisieren,
  description = {Der Begriff der Anonymisierung wird in unserem Projekt damit in Verbindung gebracht, das wir personenbezogene Objekte nach dem erhalten der Videodatei auf dem \gls{Webdienst} unkenntlich gemacht werden.},
}
\newglossaryentry{E-Mail}
{
  name = E-Mail,
  description = {Ein E-Mail ist eine briefähnliche Nachricht, die zwischen Personen mit bestimmten Netzwerkverbindungen auf einem definierten System verschickt werden kann.},
}
\newglossaryentry{REST}
{
  name = REST,
  description = {Representational State Transfer (kurz REST), beschreibt einen Programmierstil für verteilte Systeme (zum Beispiel Webservices). Die Bezeichnung leitet sich aus der Navigation zwischen Web-Seiten ab, welche man sich als Zustandsmaschine vorstellen kann.},
}
\newglossaryentry{Smartphone}
{
  name = Smartphone,
  description = {Ein Smartphone ist ein Mobilfunkgerät, welches die Funktionalität die Funktionen eines herkömmlichen Mobiltelefons überschreitet und einem tragbaren Computer ähnelt. Dabei kann man weitere Funktionalitäten via \glspl{App} auf das Endgerät herunterladen.},
}
\newglossaryentry{Push-Benarichtigungen}
{
  name = Push-Benarichtigungen,
  description = {Push-Benachrichtigungen sind Meldungen, die ohne das Öffnen der jeweiligen \gls{App} auf Ihrem \gls{Smartphone} erscheinen.},
}
\newglossaryentry{Livestream}
{
  name = Livestream,
  description = {Ein Livestream (auf deutsch Echtzeitübertragung), ist das Öffnen eines digitale Übertragungskanal. Über diesen kann ein Datenstrom, bestehend aus Video- und Audiomaterial in Echtzeit verschickt und empfangen werden. In der Regel ist das eine Übertragung von Videobildern, Fernseh-, oder Radiosendungen über das Internet.},
}
\newglossaryentry{Tablet}
{
  name = Tablet,
  description = {Ein Tablet (auch bekannt als Tablet-PC), sind größentechnisch relativ minimalistisch gehaltene Personal Computer (PC). Sie besitzen meist die gleichen Standartfunktionen wie ein normaler PC (möglicher Anschluss von Maus und Tastatur, WLAN), aber auch Funktionen die ein \gls{Smartphone} besitzt (Multitouchscreen) oder Bedienung per Stift.},
}
\newglossaryentry{iOS}
{
  name = iOS,
  description = {iOS ist ein von Apple entwickelte mobiles Betriebssystem für deren hergestellte Mobilgeräte, wie das iPhone.},
}
\newglossaryentry{WindowsPhone}
{
  name = WindowsPhone,
  description = {Windows Phone ist ein Betriebssystem für Smartphones, das von Microsoft entwickelt wurde.},
}
\newglossaryentry{Ringpuffer}
{
  name = Ringpuffer,
  description = {Der Ringpuffer (oder auch in der Informatik als Warteschlange bekannt) ist eine Datenstruktur, die allgemein zur Weiterverarbeitung von Daten verwendet wird. Wir verwenden den Ringpuffer um eine feste Zeitlänge des Videomaterials vor dem Auslösen des \gls{G-Sensor} zu ermöglichen.},
}
\newglossaryentry{RSA}
{
  name = RSA,
  description = {RSA ist ein asymmetrisches kryptographisches Verfahren, das für das Verschlüsseln und das digitale Signieren verwendet wird.},
}
\newglossaryentry{Kryptografische Verfahren}
{
  name = Kryptografische Verfahren,
  description = {Unter Kryptografie versteht man die Lehre von Geheimschriften. Das dazugehörige Verfahren wird dazu verwendet Dateien zu Verschlüsseln und diese somit vor nicht autorisierten Benutzern zu schützen.},
}
\newglossaryentry{Responsive-Design}
{
  name = Responsive-Design,
  description = {Das Responsive Webdesign stellt eine Technik zur Verfügung, um das einheitliche Anzeigen von Inhalten zu ermöglichen.},
}
\newglossaryentry{AES}
{
  name = AES,
  description = {AES (Advanced Encryption Standard), ist ein standardisiertes symmetrisches Verschlüsselungsverfahren, welches auf Blockverschlüsselung beruht.},
}


