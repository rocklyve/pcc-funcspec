\section{Funktionale Anforderungen}

\section{Web-Interface}
\begin{description}
\item[FA Anzeigen der Einlog-View] \hfill \\
Ruft der Nutzer die Privacy-Crash-Cam-Webseite auf, so gelangt er zu der Einlog-View. Dort kann sich der Benutzer anmelden (FAxx) oder sich registrieren (FAxx).

\item[FA Erstellen eines Benutzeraccounts] \hfill \\
Klickt der Benutzer auf "'Account erstellen"' so öffnet sich der Registrierungsdialog. Dort wird der Nutzer gebeten einen einzigartigen Benutzername und eine E-Mail Adresse angegeben. Zudem muss er ein Passwort auswählen und bestätigen. Klickt der Nutzer auf "'Registrierung abschließen"' werden die Eingaben überprüft. Nach dem Erstellen eines Benutzeraccounts sendet der Server eine Bestätigungsmail. Der Nutzer muss den dort enthaltenen Link klicken, um seinen Account zu verifizieren. Danach kann er sich auf der Webseite anmelden.

\item[FA Löschen eines Benutzeraccounts] \hfill \\
Klickt ein Benutzer in der Menüleiste auf "'Account Löschen"', so wird ein Bestätigungsdialog geöffnet. Bestätigt der Nutzer, so wird er ausgeloggt. Daraufhin werden alle, von ihm hochgeladenen Videos vom Server und daraufhin seine Accountdaten gelöscht.

\item[FA Einloggen auf die Webseite] \hfill \\
Zum Einloggen auf die Webseite müssen User-Name und Passwort korrekt in die entsprechenden Felder eingetragen sein. Nur verifizierte User (siehe FAxx) können sich einloggen. Ist ein Nutzer bereits angemeldet, so muss er sich zuerst in der zweiten Sitzung ausloggen, bevor er sich einloggen kann.

\item[FA Ausloggen von der Webseite] \hfill \\
Klickt ein Benutzer in der Menüleiste auf "'Ausloggen"' so wird er auf die Einlog-View zurückgeleitet. Schließt ein Nutzer die Webseite, so wird er automatisch ausgeloggt.

\item[FA Anzeigen der Menüleiste] \hfill \\
Befindet sich der Nutzer in einer anderen Ansicht als der Einlog-View, so befindet sich am linken Rand der Websteite die Menüleiste. Dort kann der Nutzer sich ausloggen (FAxx), seinen Account löschen (FAxx), die Datenschutzerklärung einsehen (FAxx) oder das Impressum einsehen (FAxx).

\item[FA Anzeigen einer Liste der anonymisierten Videos] \hfill \\
Hat sich ein Benutzer eingeloggt wird er automatisch auf diese Ansicht weitergeleitet. Hier werden die, von dem Nutzer hochgeladenen Videos chronologisch aufgelistet. Der Nutzer kann Videos herunterladen (FAxx), löschen (FAxx), ein Preview einsehen (FAxx) oder die Videoinformationen einsehen (FAxx).

\item[FA Herunterladen von anonymisierten Videos] \hfill \\
Durch einen Klick wird eine Speicherdialog geöffnet. Nachdem der Nutzer einen Speicherort ausgewählt hat wird das Video heruntergeladen.

\item[FA Löschen eines anonymisierten Videos] \hfill \\
Durch den Klick auf das "'Löschen-Symbol"' wird ein Bestätigungsdialog geöffnet. Falls der Benutzer bestätigt wird das Video aus der Liste seiner hochgeladenen Videos entfernt und vom Server gelöscht.

\item[FA Preview eines anonymisierten Videos] \hfill \\
Durch den Klick auf das "'Preview-Symbols"' wird ein Fenster geöffnet, in dem der Nutzer ein Preview des anonymisierten Videos angezeigt wird.

\item[FA Einsehen von Video-Daten der anonymisierten Videos] \hfill \\
Durch den Klick auf das "'Info-Symbol"' wird ein Fenster geöffnet, dass dem Benutzer die Video-Metadaten (Erstellungsdatum, Datum der Anonymisierung, Größe, Auflösung, Dauer) anzeigt.

\item[FA Anzeigen der Datenschutzerklärung] \hfill \\
Klickt ein Benutzer in der Menüleiste auf "Datenschutz", so wird eine Sicht geöffnet, in der der Nutzer die Datenschutzerklärung und das AGB einsehen kann.

\item[FA Anzeigen des Impressums] \hfill \\
Klickt ein Benutzer in der Menüleiste auf "Impressum", so wird eine Sicht geöffnet, in der der Nutzer das Impressum einsehen kann.

\end{description}

\section{Web-Service}
\begin{description}
\item[FA Empfangen eines Videos von der App] \hfill \\
Bekommt der Web-Service eine Anfrage von der App ein Video hochzuladen, so überprüft er zunächst, ob er die Anfrage bearbeiten kann, oder ob bereits zu viele andere Anfragen gestellt wurden (NAxx). Ist dies nicht der Fall, so speichert er das Video temporär und beginnt die Anonymisierung (FAxx-FAxx).

\item[FA Entschlüsseln eines empfangenen Videos] \hfill \\
Bevor der Web-Service die Bearbeitung des Videos beginnt, entschlüsselt er das empfangene verschlüsselte Video. Das entschlüsselte Video wird lokal temporär gespeichert.

\item[FA Schätzen der Bearbeitungszeit] \hfill \\
Nachdem das Video entschlüsselt wurde schätzt der Web-Service anhand der Länge und Auflösung des Videos den Bearbeitungsaufwand  ab und übermittelt eine Schätzzeit an die App.

\item[FA Identifizieren der relevanten Bildbereiche] \hfill \\
Der Web-Service nimmt das entschlüsselte Video und lässt einen Bildfilter über das Video laufen, der die für die Anonymisierung relevanten Bildbereiche (Gesichter, Nummernschilder, etc.) erkennt. Die so ermittelteten Bereiche werden in einer Bitmaske gespeichert.

\item[NA Anonymisierung des Videos] \hfill \\
Der Web-Service nimmt die in FAxx erstellte Bitmaske um die dort makierten relevanten Bildbereiche mithilfe eines Anonymisierungsfilters zu anonymisieren.

\item[FA Abspeichern eines anonymisierten Videos] \hfill \\
Nachdem das Video anonymisiert wurde, wird es lokal auf dem Server gespeichert und alle temporären Dateien gelöscht. Das gespeicherte Video wird der Videoverwaltung hinzugefügt damit es vom Benutzer eingesehen und bearbeitet werden kann.
\end{description}

\section{App}
\begin{description}
\item[FA Starten der App] \hfill \\

\item[FA Einloggen in einen Benutzeraccount] \hfill \\

\item[FA Ausloggen von einem Benutzeraccount] \hfill \\

\item[FA Anzeigen der Vorschau im "Beobachtungsmodus"] \hfill \\

\item[FA Beschreiben des Ringpuffers] \hfill \\

\item[FA Auslösen des G-Sensors] \hfill \\

\item[FA Auslösen der manuellen Aufnahmefunktion] \hfill \\

\item[FA Speichern eines Videos] \hfill \\
Vor dem Speichern des Videos werden relevante Metadaten eingefügt (Ort, Zeit) und das Video verschlüsselt.

\item[FA Anzeigen des Menüs] \hfill \\

\item[FA Navigieren im Menü] \hfill \\

\item[FA Anzeigen der Einstellungen] \hfill \\

\item[FA Anzeigen der Datenschutzerklärung] \hfill \\

\item[FA Bearbeiten von Einstellungen] \hfill \\

\item[FA Anzeigen der gespeicherten Videos]

\item[FA Löschen von gespeicherten Videos]

\item[FA Anzeigen einer Benachrichtigung zum Löschen von Videos]

\item[FA Hochladen von gespeicherten Videos]

\item[FA Anzeigen der Meta-Daten von gespeicherten Videos]

\item[FA Schließen der App]
\end{description}