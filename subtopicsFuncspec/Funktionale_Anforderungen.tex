\chapter{Funktionale Anforderungen}

\section{App}
\begin{enumerate}
\renewcommand{\labelenumi}{\textbf{\theenumi}}
\renewcommand{\theenumi}{FA\arabic{enumi}0}
\setcounter{enumi}{99}
\item \textbf{Anzeigen der Einlog-Ansicht} \hfill \\
Öffnet der Benutzer die App und sind keine Nutzerdaten gespeichert, so gelangt er in die Einlog-Ansicht. Dort kann er sich einloggen. Das Erstellen von Benutzeraccounts ist hier \textbf{nicht} möglich.

\item \label{fa:Anmeldedaten}\textbf{Anmeldedaten} \hfill \\
Anmeldedaten bestehen aus der E-Mai-Adresse und dem Passwort des Nutzers. Anemeldedaten werden lokal auf dem Gerät abgespeichert.

\item \textbf{Einloggen in die App} \hfill \\
Beim Appstart muss sich der Nutzer zunächst einloggen. Zum Einloggen in die App müssen die Anmeldedaten~\eqref{fa:Anmeldedaten} des Nutzers korrekt in die entsprechenden Felder eingetragen sein. Nur verifizierte Nutzer ~\eqref{fa:erstellAcc}) können sich einloggen. Beim Einloggen werden Nutzername und Passwort durch eine Serveranfrage geprüft. Bei falschen Eingaben oder wenn der Nutzer seinen Account nicht verifiziert hat erhält der Nutzer eine Fehlermeldung. Hat sich ein Nutzer bereits zuvor angemeldet, ohne sich wieder abzumelden, gelangt der Nutzer direkt zur Kamera-Ansicht. Die überprüfung der gespeicherten Anmeldedaten~\eqref{fa:Anmeldedaten} erfolgt dann erst beim Hochladen verschlüsselter Videodaten auf den Server~\eqref{fa:vidHochladen}.

\item \label{fa:logOut}\textbf{Ausloggen von einem Benutzeraccount} \hfill \\
Klickt ein Benutzer im Menü auf ``ausloggen'', so wird er auf die Einlog-Ansicht zurückgeleitet und seine Anemldedaten~\eqref{fa:Anmeldedaten} vom Gerät gelöscht.

\item \label{fa:Beobachtungsmodus}\textbf{Ausführen des Beobachtungsmodus} \hfill \\
Die Kamera kann sich bei aktiver Kamera in zwei Modi befinden: Im Beobachtungsmodus~\eqref{fa:Beobachtungsmodus} oder im Aufnahmemodus~\eqref{fa:Aufnahmemodus}.
Im Beobachtungsmodus werden Kamerabilder nicht persistiert sondern nur der Ringpuffer~\eqref{fa:Ringpuffer} beschrieben. Die G-Sensordaten werden ausgewertet und es wird auf einen charakteristischen Ausschlag des G-Sensors gewartet.

\item \label{fa:Statussymbol}\textbf{Anzeigen des Statussymbols} \hfill \\
Der aktuelle Kameramodus wird dem Nutzer durche ein blinkendes Statussymbol am Bildschirmrand visualisiert. Im Beobachtungsmodus hat es eine grüne Farbe, im Aufnahmemodus hat es eine rote Farbe.

\item \label{fa:camStart}\textbf{Starten der Beobachtung} \hfill \\
Wird die Kamera-Ansicht gestartet, befindet sich die Kamera sofort im Beobachtungsmodus. Nun wird der Ringpuffer~\eqref{fa:Ringpuffer} beschrieben.

\item \textbf{Stoppen der Beobachtung} \hfill \\
Wenn der Benutzer die Kamera-Ansicht während der Beobachtung verlässt, oder die App schließt, wird die Beobachtung automatisch beendet.

\item \label{fa:automUebergang}\textbf{Durch G-Sensor ausgelöster Übergang in den Aufnahmemodus} \hfill \\
Werden die in ~\eqref{na:GSensfront} bis ~\eqref{na:GSensvert} definierten Richtwerte des G-Sensors überschritten, während die App die Kamera-Ansicht anzeigt, geht die Kamera in den Aufnahmemodus über~\eqref{fa:Aufnahmemodus}. 

\item \label{fa:manUebergang}\textbf{Manueller Übergang in den Aufnahmemodus} \hfill \\
Befindet sich die Kamera im Beobachtungsmodus~\eqref{fa:Beobachtungsmodus} geht sie nach doppeltem Tippen auf die Vorschaufläche der Kamera-Ansicht in den Aufnahmemodus über.

\item \label{fa:Aufnahmemodus}\textbf{Ausführen des Aufnahmemodus} \hfill \\
Die Kamera kann sich bei aktiver Kamera in zwei Modi befinden: Im Beobachtungsmodus~\eqref{fa:Beobachtungsmodus} oder im Aufnahmemodus~\eqref{fa:Aufnahmemodus}.
Im Aufnahmemodus wird der Ringpuffers~\eqref{fa:Ringpuffer} für weitere 30 Sekunden beschrieben und anschließend dessen Inhalt im Hintergrund verschlüsselt~\eqref{fa:Verschluesselung} und persistiert. Die G-Sensordaten sowie Nutzereingaben werden während sich die Kamera im Aufnahmemodus befindet ignoriert. Die Messwerte des G-Sensors, Zeit und Auslöseart(~\eqref{fa:manUebergang}, ~\eqref{fa:automUebergang}) werden in die Metadaten der Videodatei geschrieben. Nach Ablauf der erwähnten 30 Sekunden wechselt die Kamera wieder zurück in den Beobachtungsmodus~\eqref{fa:Beobachtungsmodus}.

\item \label{fa:Ringpuffer}\textbf{Ringpuffer beschreiben} \hfill \\
Der Ringpuffer~\eqref{fa:Ringpuffer} stellt einen temporären Speicher dar, in den die Bilder der Kamera unverschlüsselt geschrieben werden. Dabei verhält er sich wie eine Warteschlange und fasst eine Minute Videomaterial, die Tonspur wird verworfen. Die später persistierte Videodatei soll 30 Sekunden vor dem Übergang der Kamera in den Aufnahmemodus und 30 Sekunden nach dem Übergang enthalten, wobei sich der Übergang in der Mitte des Videos befinden soll.

\item \label{fa:Verschluesselung}\textbf{Verschlüsseln eines Videos} \hfill \\
Videodaten werden durch das unter Kaptiel 9.4 defininierte hybrides Verschlüsselungsverfahren verschlüsselt.

\item \textbf{Anzeigen des Menüs} \hfill \\
Drückt der Benutzer den "'Menü-Button"' in der oberen linken Ecke des Bildschirms, so öffnet sich das Menü. befindet sich die Kamrea im Beobachtungsmodus, während das Menü geöffnet wird, wird die Aufnahme nicht gestoppt. In dem Menü hat der Benutzer die Möglichkeit zwischen den verschiedenen App-Ansichten Video-Ansicht~\eqref{fa:vidAnsicht}, Einstellungs-Ansicht~\eqref{fa:einstAnsicht} und Impressum-Ansicht~\eqref{fa:imprAnsicht} oder sich auszuloggen~\eqref{fa:logOut}.

\item \label{fa:vidAnsicht}\textbf{Anzeigen der Liste der persistierten Videos} \hfill \\
Wählt der Benutzer im Menü die Option "'Meine Videos"', so gelangt er zu einer Ansicht, in dem ihm seine persistierten Videos chronologisch aufgelistet werden. Der Nutzer kann Videos hochladen~\eqref{fa:vidHochladen}, löschen~\eqref{fa:vidLöschen}, oder Videoinformationen einsehen~\eqref{fa:metaVerschlVid}.

\item \label{fa:vidHochladen}\textbf{Hochladen von gespeicherten Videos} \hfill \\
Klickt der Nutzer auf den ``Upload-Button'', wird ein Dialog aufgerufen, in dem der Nutzer darauf hingewiesen wird, dass mobile Daten anfallen werden. Wenn der Benutzer akzeptiert, schickt die App eine Anfrage an den Server. Dieser beantwortet die Anfrage falls Ressourcen verfügbar und die auf dem Gerät gespeicherten Anmeldedaten des Nutzers korrekt sind und erlaubt den Upload. In jedem Fall wird dem Nutzer eine Erfolgs- bzw. Misserfolgsnachricht gezeigt. Durch Tippen auf diese kann er zur Liste seiner Videos~\eqref{fa:vidAnsicht} gelangen.

\item \label{fa:vidLöschen}\textbf{Löschen von gespeicherten Videos} \hfill \\
Klickt der Benutzer auf das ``Löschen-Symbol'', so wird ein Bestätigungsdialog~\eqref{fa:vidLöschenDialog} geöffnet. Falls der Benutzer bestätigt wird das Video aus der Liste seiner persistierten Videos entfernt und vom Gerät gelöscht. Bricht der Benutzer den Dialog ab bleibt er in der Listenansicht seiner Videos.

\item \label{fa:vidLöschenDialog}\textbf{Anzeigen einer Benachrichtigung zum Löschen von Videos} \hfill \\
Beim Einloggen und bei jedem Appstart wird geprüft, ob persistierte Videos bereits seit über 4 Wochen auf seinem Gerät gespeichert sind. Ist dies so wird ihm ein Dialog angezeigt, der ihn auf diesen Umstand hinweist. Dort wird ihm angeboten, das Video zu löschen~\eqref{fa:vidLöschen}. Bricht er den Dialog ab, gelangt er wie üblich in die Kamera-Ansicht~\eqref{fa:camStart}.

\item \label{fa:metaVerschlVid}\textbf{Einsehen von Video-Daten der anonymisierten Videos} \hfill \\
Klickt der Benutzer das ``Info-Symbol'' wird ein Fenster geöffnet, dass dem Benutzer die Video-Metadaten (Erstellungsdatum, Größe, Auflösung, Dauer, Auslöseart, G-Sensor-Daten) als Dialog anzeigt. Schließt der Nutzer den Dialog, kehrt er zu der Liste seiner Videos~\eqref{fa:vidAnsicht} zurück.

\item \label{fa:einstAnsicht}\textbf{Anzeigen der Einstellungen}
Wählt der Benutzer im Menü die Option ``Einstellungen'', so werden dem Nutzer die Standardeinstellungen angezeigt (Auflösung, Bildwiederholrate, Größe Ringpuffer) angezeigt.

\item \label{fa:imprAnsicht}\textbf{Anzeigen rechtlicher Informationen} \hfill \\
Wählt der Benutzer im Menü die Option ``Impressum'', gelangt er zur Impressums-Ansicht. Von dort kann er sich das Impressum~\eqref{fa:imprAnzeigen} und die Datenschutzerklärung~\eqref{fa:datenschAnzeigen} anzeigen lassen.

\item \label{fa:imprAnzeigen}\textbf{Anzeigen des Impressums} \hfill \\
Wählt der Benutzer ``Impressum'' auf der Impressum-Ansicht, wird ein Dialog angezeigt, der das Impressum anzeigt.

\item  \label{fa:datenschAnzeigen}\textbf{Anzeigen der Datenschutzerklärung} \hfill \\
Wählt der Benutzer ``Datenschutzerklärung'' auf der Impressum-Ansicht, wird ein Dialog angezeigt, der die Datenschutzerklärung anzeigt.

\end{enumerate}

\section{Web-Dienst}
\begin{enumerate}
\renewcommand{\labelenumi}{\textbf{\theenumi}}
\renewcommand{\theenumi}{FA\arabic{enumi}0}
\setcounter{enumi}{199}

\item  \textbf{Empfangen eines Videos von der App} \hfill \\
Bekommt der Web-Service eine Anfrage von der App ein Video hochzuladen, so überprüft er zunächst, ob er die Anfrage bearbeiten kann, oder ob bereits zu viele andere Anfragen gestellt wurden ~\eqref{na:paralleleZugriffe}. Ist dies nicht der Fall, so speichert er das Video temporär~\eqref{fa:entschVideo} und beginnt die Anonymisierung~\eqref{fa:anonymVideo}.

\item  \label{fa:entschVideo}\textbf{Entschlüsseln eines empfangenen Videos} \hfill \\
Bevor der Web-Service die Bearbeitung des Videos beginnt, entschlüsselt er das empfangene verschlüsselte Video. Hierbei wird das unter Kaptiel 9.4 defininierte Verschlüsselungsverfahren angewandt. Das entschlüsselte Video wird lokal temporär gespeichert.

\item  \label{fa:relBildbereiche}\textbf{Identifizieren der relevanten Bildbereiche} \hfill \\
Der Web-Service nimmt das entschlüsselte Video und lässt einen Bildfilter über das Video laufen, der die für die Anonymisierung relevanten Bildbereiche (Gesichter, Nummernschilder, etc.) erkennt. Die so ermittelteten Bereiche werden in einer Bitmaske gespeichert.

\item  \label{fa:anonymVideo}\textbf{Anonymisierung des Videos} \hfill \\
Der Web-Service nimmt die in ~\eqref{fa:relBildbereiche} erstellte Bitmaske um die dort makierten relevanten Bildbereiche mithilfe eines Anonymisierungsfilters zu anonymisieren.

\item \label{fa:speichVideo}\textbf{Abspeichern eines anonymisierten Videos} \hfill \\
Nachdem das Video anonymisiert wurde, wird es lokal auf dem Server gespeichert und alle temporären Dateien gelöscht. Das gespeicherte Video wird der Videoverwaltung hinzugefügt damit es vom Benutzer eingesehen und bearbeitet werden kann. Wenn ein Benutzer die maximale Anzahl Videos pro Account ~\eqref{na:VideoKap} überschreitet, wir automatisch das älteste Video des Accounts auf dem Server gelöscht.
\end{enumerate}

\section{Web-Interface}
\begin{enumerate}
\renewcommand{\labelenumi}{\textbf{\theenumi}}
\renewcommand{\theenumi}{FA\arabic{enumi}0}
\setcounter{enumi}{299}
\item \textbf{Anzeigen der Einlog-Ansicht} \hfill \\
Ruft der Nutzer die Privacy-Crash-Cam-Webseite auf, so gelangt er zu der Einlog-Ansicht. Dort kann sich der Benutzer anmelden ~\eqref{fa:weblogIn} oder sich registrieren (~\eqref{fa:erstellAcc}.

\item \label{fa:erstellAcc}\textbf{Erstellen eines Benutzeraccount} \hfill \\
Klickt der Benutzer auf "'Account erstellen"' so öffnet sich der Registrierungsdialog. Dort wird der Nutzer gebeten einen einzigartigen Benutzername und eine E-Mail Adresse angegeben. Zudem muss er ein Passwort auswählen und bestätigen. Klickt der Nutzer auf "'Registrierung abschließen"' werden die Eingaben überprüft. Schlägt dies fehl bleibt der Benutzer in dem Registrierungsdialog. Nach dem Erstellen eines Benutzeraccounts sendet der Server eine Bestätigungsmail. Der Nutzer muss den dort enthaltenen Link klicken, um seinen Account zu verifizieren. Danach kann er sich auf der Webseite anmelden.

\item \label{fa:löschAcc}\textbf{Löschen eines Benutzeraccounts} \hfill \\
Klickt ein Benutzer in der Menüleiste auf "'Account Löschen"', so wird ein Bestätigungsdialog geöffnet. Bestätigt der Nutzer, so wird er ausgeloggt. Daraufhin werden alle, von ihm hochgeladenen Videos vom Server und daraufhin seine Accountdaten gelöscht.

\item \label{fa:weblogIn}\textbf{Einloggen auf die Webseite} \hfill \\
Zum Einloggen auf die Webseite müssen Benutzername und Passwort korrekt in die entsprechenden Felder eingetragen sein. Nur verifizierte User ~\eqref{fa:erstellAcc} können sich einloggen. Ist ein Nutzer bereits angemeldet, so muss er sich zuerst in der zweiten Web-Sitzung ausloggen, bevor er sich einloggen kann. Bei falschen Eingaben oder wenn der Nutzer bereits eingeloggt ist kehrt er zur Einlog-Ansicht zurück und erhält eine Fehlermeldung.

\item \label{fa:weblogOut}\textbf{Ausloggen von der Webseite} \hfill \\
Klickt ein Benutzer in der Menüleiste auf "'Ausloggen"' so wird er auf die Einlog-Ansicht zurückgeleitet. Schließt ein Nutzer die Webseite, so wird er automatisch ausgeloggt.

\item \textbf{Anzeigen der Menüleiste} \hfill \\
Befindet sich der Nutzer in einer anderen Ansicht als der Einlog-Ansicht, so befindet sich am linken Rand der Websteite die Menüleiste. Dort kann der Nutzer die Liste der anonymisierten Videos ~\eqref{fa:anonymVidAnzeigen}, Datenschutzerklärung einsehen ~\eqref{fa:datenschAnzeigen}, das Impressum einsehen ~\eqref{fa:imprAnzeigen}, sich ausloggen ~\eqref{fa:weblogOut}, seinen Account löschen ~\eqref{fa:löschAcc}.

\item \label{fa:anonymVidAnzeigen}\textbf{Anzeigen der Liste der anonymisierten Videos} \hfill \\
Hat sich ein Benutzer eingeloggt wird er automatisch auf diese Ansicht weitergeleitet. Hier werden die, von dem Nutzer hochgeladenen Videos chronologisch aufgelistet. Der Nutzer kann Videos herunterladen ~\eqref{fa:anonymVidherunt}, löschen ~\eqref{fa:anonymVidlösch}, ein Preview einsehen ~\eqref{fa:anonymVidprev} oder die Videoinformationen einsehen ~\eqref{fa:anonymViddaten}.

\item \label{fa:anonymVidherunt}\textbf{Herunterladen von anonymisierten Videos} \hfill \\
Durch einen Klick wird eine Speicherdialog geöffnet. Nachdem der Nutzer einen Speicherort ausgewählt hat wird das Video heruntergeladen. Bricht der Benutzer den Dialog ab bleibt er in der Listenansicht seiner Videos.

\item \label{fa:anonymVidlösch}\textbf{Löschen eines anonymisierten Videos} \hfill \\
Durch den Klick auf das "'Löschen-Symbol"' wird ein Bestätigungsdialog geöffnet. Falls der Benutzer bestätigt wird das Video aus der Liste seiner hochgeladenen Videos entfernt und vom Server gelöscht. Bricht der Benutzer den Dialog ab bleibt er in der Listenansicht seiner Videos.

\item \label{fa:anonymVidprev}\textbf{Vorschau eines anonymisierten Videos} \hfill \\
Klickt der Benutzer auf das "'Vorschau-Symbol"', so wird ein Fenster geöffnet, in dem der Nutzer ein Vorschau des anonymisierten Videos angezeigt wird.

\item \label{fa:anonymViddaten}\textbf{Einsehen von Video-Daten der anonymisierten Videos} \hfill \\
Klickt der Benutzer auf das "'Info-Symbol"', so wird ein Fenster geöffnet, dass dem Benutzer die Video-Metadaten (Erstellungsdatum, Datum der Anonymisierung, Größe, Auflösung, Dauer) anzeigt.

\item \textbf{Anzeigen der Datenschutzerklärung} \hfill \\
Klickt der Benutzer in der Menüleiste auf "'Datenschutz"', so wird eine Sicht geöffnet, in der der Nutzer die Datenschutzerklärung und die AGB einsehen kann.

\item \textbf{Anzeigen des Impressums} \hfill \\
Klickt der Benutzer in der Menüleiste auf "'Impressum"', so wird eine Sicht geöffnet, in der der Nutzer das Impressum einsehen kann.
\end{enumerate}