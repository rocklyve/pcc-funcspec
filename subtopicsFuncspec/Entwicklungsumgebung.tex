\chapter{Entwicklungsumgebung}
\section{Beschreibung}
Die Programmiersprache mit der wir unser Projekt realisieren ist Java. Dafür verwenden wir das Java Web UI Framework Vaddin, welche eine serverseitige Architektur anbietet. Vaddin benutzt das Google Web Toolkit zur Darstellung von Webseiten, mit der wir unsere Website erstellen. Eclipse oder Intelij, beides integrierte Entwicklungsumgebungen, werden für die Implementierung genutzt. Für die Android Applikation verwenden wir die frei integrierte Entwicklungsumgebung Android Studio, welche sich auf die Intelij IDEA stützt.\newline
\linebreak
Zur Aufgabenverteilung verwenden wir das von Atlassian entwickelte Projektmanagement-Tool JIRA. Dies wird zur Projekt- und Vorgangsverfolgung unserer Arbeit benutzt. 
Um eine Versionskontrolle aller Dokumente zu ermöglichen benutzen wir GIT um nicht-lineare Entwicklung durchzuführen. 
Der Web-Service GitHub ermöglicht uns unsere Repositories zu verwalten. \newline
\linebreak
Zur Erstellung von UML-Diagrammen wurde das freie unabhängige UML-Werkzeug UMLet verwendet.  Mit diesem wurden Anwendungs- und Aktivitätsdiagramme erstellt, um bestimmte Bedienung und Abläufe zu veranschaulichen.
\section{Zusammenfassung}