\chapter{Produktdaten}
Zur Verwendung des Produktes werden vom jeweiligen Teil der Software Daten angefordert und somit auch abgelegt, die jedoch haupts\"achlich auf der Datenbank des Servers gespeichert sind.
\section{App-Daten}
\begin{description}
\item[PD Kundendaten]\hfill \\
Es werden Nachname, Vorname sowie die E-Mail Adresse des Kunden abglegt.
\item[PD Einstellungen]\hfill \\
Es werden Einstellungen der Kamera, sprich die Auflösung der Aufnahme gespeichert.
\item[PD Ringpuffer]\hfill \\
Es werden eine Minute an Video Material im Ringpuffer gespeichert.
\item[PD Videodaten]\hfill \\
Es werden Videodaten, sowie Ort und Zeit in die Meta-Daten des Videos gespeichert.
\end{description}

\section{Web-Service}
\begin{description}
\item[PD Videodaten]\hfill \\
Es werden Temporär Videodaten, sowie Ort und Zeit in den Meta-Daten des Videos gespeichert.
\item[PD Kundendaten]\hfill \\
Es werden Nachname, Vorname sowie die E-Mail Adresse des Kunden in einer Datenbank gespeichert. Jeder Kunde bekommt eine einzigartige ID zugewiesen. 
\end{description}

\section{Webinterface Daten}
\begin{description}
\item[PD Daten]\hfill \\
Es werden keine Daten auf dem Webinterface gespeichert. Alle Informationen werden mit REST-Anfragen vom Webservice abgerufen.
\end{description}


